\begin{frame}{Introdution (1): Document-oriented databases}
  \begin{itemize}
    \item document-oriented databases
    \begin{itemize}
      \item stores documents
      \item create, read, update and delete documents
    \end{itemize}
    \item documents do \textbf{not} have to be of the same shape
    \begin{itemize}
      \item oriented on the data itself, not the shape
      \item flexible w.r.t. data inside them
        \inote{missing / optional values easy to incoperate}
      \item easy to change format of data on-demand
      \item referential integrity (if applicable) difficult
    \end{itemize}
  \end{itemize}
\end{frame}

\begin{frame}[fragile]{Introdution (2): Structure of MongoDB}
  \begin{itemize}
    \item \logoimage{imgs/logo}{20px} is an open-source, document database designed for ease of development and scaling
    \begin{itemize}
      \item{so it is document-oriented}
    \end{itemize}
    \item stores documents so-called ``records''
    \begin{itemize}
      \item documents are essentially JSON, i.e. key-value pairs
      \item keys are strings
      \item values can for example be strings, numbers, other documents, arrays of values, arrays of other documents
    \end{itemize}
  \end{itemize}
\end{frame}

\begin{frame}[fragile]{Introdution (3): Structure of MongoDB}
  \begin{tabular}{c}
    \begin{lstlisting}
{
  "_id" : ObjectId("54c955492b7c8eb21818bd09"),
  "address" : {
    "street" : "2 Avenue",
    "zipcode" : "10075",
    "building" : "1480",
    "coord" : [ -73.9557413, 40.7720266 ],
  },
  "borough" : "Manhattan",
  "cuisine" : "Italian",
  "grades" : [
    {
      "date" : ISODate("2014-10-01T00:00:00Z"),
      "grade" : "A",
      "score" : 11
    },
    {
      "date" : ISODate("2014-01-16T00:00:00Z"),
      "grade" : "B",
      "score" : 17
    }
  ],
  "name" : "Vella",
  "restaurant_id" : "41704620"
}
    \end{lstlisting}
  \end{tabular}
\end{frame}

\begin{frame}[fragile]{Introdution (4): Structure of MongoDB}
  \centering
  \begin{itemize}
    \item each mongodb server can host several databases
      \inote{like in MySQL}
    \item each database can host several collections
      \inote{similar to tables}
    \item each collection contains a set of documents
      \inote{\textbf{but:} documents do not need to be of the same shape}
  \end{itemize}
\end{frame}
